\section{Einleitung}

Diese Bachelorarbeit entsteht in Zusammenarbeit mit der Pixelhouse GmbH. Diese betreibt seit ca. 10 Jahren die Webseite Chefkoch.de, Europas größtes Kochportal. Neben der Vermarktung der Webseite und der Betreuung der Community wird von den Mitarbeitern vor allem auch die softwaretechnische Entwicklung der Plattform und das Hosting durchgeführt \citep[Vgl.][]{pixelhouse14}.

Um besser auf zukünftige Herausforderungen reagieren zu können, hat die Pixelhouse GmbH im letzten Jahr mit der Implementierung einer neuen IT Strategie begonnen. Teil dieser Strategie ist das Einführen von Scrum als agile Softwareentwicklungsmethode. "`Im Kern einer jeden agilen Methode stehen Iterationen, die Zyklen fixer Dauer sind und zum Ziel haben, Feedback zu liefern. In iterativen Vorgehensweisen werden sämtliche Aktivitäten, die wir aus klassischen Vorgehensmodellen kennen, wie Analyse, Design, Implementierung, Testen, Integrieren, Systemtest etc., durchgeführt, jedoch mehrmals."' \citep[S.][S. 18]{wintersteiger13}

Zur Optimierung dieser Feedbackschleifen setzt die Pixelhouse GmbH wie auch andere Firmen auf die von Paul M. Duvall in seinem Buch \ac{CI} beschriebene Methode der kontinuierlichen Integration und die dazugehörigen Werkzeuge \citep[Vgl.][S. 12]{DuvMatAnd07}: So wird jede Änderung an der Software in ein Versionskontrollsystem eingespielt, welches daraufhin einen automatischen Prozess startet, der im Rahmen eines sogenannte Build-Scripts unter anderem automatische Softwaretests durchführt und die entsprechenden Ergebnisse per E-Mail an die Entwickler zurückliefert.

Beim Ausbau dieses Prozesses ist die Pixelhouse GmbH nun auf einige Probleme gestoßen, die im Rahmen dieser Bachelorarbeit allgemein diskutiert und einer Lösung zugeführt werden sollen.

\subsection{Problemstellung}

Das größte Problem mit dem zuvor beschriebenen Prozess ist aktuell, dass die automatischen Softwaretests nicht skalieren. So werden unter anderem Tests durchgeführt, die mit Hilfe einer programmatischen Schnittstelle normale Webbrowser fernsteuern und so die Nutzung der Webseite durch echte Nutzer simulieren. Diese Art von Test sind sehr langsam und die aktuell existierenden Tests benötigen bereits mehrere Stunden zur Durchführung. Dies bedeutet, dass man nach fertiger Implementierung einer neuen Funktion oder der Modernisierung einer bestehenden Komponente sehr lange auf das Testergebnis warten muss und so Zeit vergeht, bis sich die Änderung sicher in den Hauptentzwicklungszweig integrieren oder in Produktion nehmen lässt. Dies stört den Wunsch nach häufigen Iterationen und schnellem Feedback.

Eine mögliche Lösung für dieses Problem wird von Jez Humble und David Farley in ihrem Buch \ac{CD} beschrieben: "`Assuming that your tests are all independent [..], you can run them in parallel [..]. Ultimately, the performance of your tests is only limited by the time it takes for your single slowest test case to run and the size of your hardware budget."' \citep[S.][S. 310]{HumFar10} Beim Versuch, eine entsprechende Parallelisierung der Tests zu erreichen, scheitert die Pixelhouse GmbH bislang daran, entsprechend isolierte Testumgebungen vorzuhalten. So laufen die Tests bisher eher in der selben Umgebung, zum Beispiel auf dem gleichen Datenbankserver und hinter dem gleichen HTTP-Cache. Dadurch kann es vorkommen, dass ein Test die Daten oder das Caching eines anderen Tests beinflusst und ihn somit fehlschlagen lässt. Das Aufsetzen von Umgebungen ist bislang ein manueller Prozess. "`It is extremely difficult to precisley reproduce manually configured environments for testing purposes."' \citep[S.][S. 49]{HumFar10}

Als mögliche Lösung für dieses Problem empfehlen Jez Humble und David Farley den Einsatz von Virtualisierungslösungen. "`The use of virtual servers to baseline host environments makes it simple to create copies of production environments, even where a production environment consists of serveral servers, and to reproduce them for testing purposes."' \citep[S.][S. 304]{HumFar10}

\subsection{Ziel}

Ziel dieser Bachelorarbeit ist es zu untersuchen, ob und wie man mit Hilfe von Virtualisierungstechniken eine Lösung für das zuvor beschriebene Problem langsamer Feedbackschleifen bei der Ausführung automatischer Softwaretests schaffen kann. Mit dieser Lösung soll nicht nur die Anwendung sondern auch deren Infrastruktur in einem konkreten Zustand nachgehalten und effizient aufgesetzt werden können. Dadurch soll es möglich sein, einfach und beliebig oft Testumgebungen anbieten zu können, um die Ausführung der Tests parallelliesieren zu können und so die Gesamtlaufzeit der Tests zu reduzieren. Neben der konzeptionellen Arbeit sollen auch prototypische Umsetzungen erfolgen und diese anschließend bewertet werden.

\subsection{Vorgehensweise}

Diese Arbeit gibt im zweiten Kapitel "`Grundlagen und Ansätze"' zunächst einen theoretischen Überblick über die für die Problemstellung relevanten Themenbereiche Softwaretests und Virtualisierung. Im darauf folgenden Kapitel "`Konzeption"' soll dann aufgrund dieser theoretischen Ansätze erarbeitet werden, wie eine mögliche Lösung für die vorliegende Problemstellung aussieht. Im anschließenden Kapitel "`Implementierung"' werden die Ergebnisse verschiedener prototypischer Umsetzungen - vor allem deren Probleme und Besonderheiten - beschrieben. Im vorletzten Kapitel "`Evaluation"' soll eine objektive Bewertung der verschiedenen Umsetzungen anhand bestimmter Kriterien erfolgen. Ein wichtiges Kriterium ist auf jeden Fall die Reduzierung der Gesamtausführungsdauer der Tests, aber zum Beispiel auch die einfache Verwendbarkeit der Lösung. Andere Kriterien wie zum Beispiel monetäre Kosten oder die Virtualisierbarkeit anderer Umgebungen (Windows, MacOS) fallen hingegen weniger ins Gewicht. Das letzte Kapitel "`Zusammenfassung und Ausblick"' fasst die Ergebnisse der Arbeit zusammen und gibt einen kurzen Ausblick auf mögliche weitere Schritte.

